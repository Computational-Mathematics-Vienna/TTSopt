\documentclass[]{article}
\usepackage[a4paper, left=3cm, right=3cm, top=2cm]{geometry}

\usepackage{lmodern}
\usepackage{amssymb,amsmath}
\usepackage{ifxetex,ifluatex}
\usepackage{fixltx2e} % provides \textsubscript
\ifnum 0\ifxetex 1\fi\ifluatex 1\fi=0 % if pdftex
  \usepackage[T1]{fontenc}
  \usepackage[utf8]{inputenc}
\else % if luatex or xelatex
  \ifxetex
    \usepackage{mathspec}
  \else
    \usepackage{fontspec}
  \fi
  \defaultfontfeatures{Ligatures=TeX,Scale=MatchLowercase}
\fi
% use upquote if available, for straight quotes in verbatim environments
\IfFileExists{upquote.sty}{\usepackage{upquote}}{}
% use microtype if available
\IfFileExists{microtype.sty}{%
\usepackage[]{microtype}
\UseMicrotypeSet[protrusion]{basicmath} % disable protrusion for tt fonts
}{}
\PassOptionsToPackage{hyphens}{url} % url is loaded by hyperref
\usepackage[unicode=true]{hyperref}
\hypersetup{
            pdfborder={0 0 0},
            breaklinks=true}
\urlstyle{same}  % don't use monospace font for urls
\IfFileExists{parskip.sty}{%
\usepackage{parskip}
}{% else
\setlength{\parindent}{0pt}
\setlength{\parskip}{6pt plus 2pt minus 1pt}
}
\setlength{\emergencystretch}{3em}  % prevent overfull lines
\providecommand{\tightlist}{%
  \setlength{\itemsep}{0pt}\setlength{\parskip}{0pt}}
\setcounter{secnumdepth}{0}
% Redefines (sub)paragraphs to behave more like sections
\ifx\paragraph\undefined\else
\let\oldparagraph\paragraph
\renewcommand{\paragraph}[1]{\oldparagraph{#1}\mbox{}}
\fi
\ifx\subparagraph\undefined\else
\let\oldsubparagraph\subparagraph
\renewcommand{\subparagraph}[1]{\oldsubparagraph{#1}\mbox{}}
\fi

% set default figure placement to htbp
\makeatletter
\def\fps@figure{htbp}
\makeatother


\date{}

\begin{document}

\begin{enumerate}

\item
  To write a wrapper and/or tuner for a solver, make a new python file
  (version \textgreater{}= 3.6, tested on 3.9) and import the
  \texttt{Solver} and \texttt{Parameter} classes from the generator.
  Make sure that you use the correct path, e.g.~if you're in the
  \texttt{scripts} folder then use \texttt{from\ generator.generator}.
\item
  Specify tuning parameters by creating a list with instances of the
  \texttt{Parameter} class using the following arguments:
\end{enumerate}

\begin{verbatim}
name : str
  Required argument that specifies the name of the parameter.

value : float | int | str
  Required argument with a predefined value of the parameter.

tuning_range : list(float | int | str)
  Optional argument that specifies the interval of continuous or integer
  tuning variables or all the options of categorical ones.
  Defaults to empty list.

input_type : type
  Optional argument that specifies what kind of values the
  parameter attains.
  Must be `float`, `int` or `str`.
  Defaults to type of given `value`.

optimization_type : str
  Optional argument that specifies how the variable is treated
  when tuning.
  Must be `"continuous"`, `"integer"` or `"categorical"`.
  Default depends on type of given `value` as follows:
  `float` -> `"continuous"`,
  `int` -> `"integer"`,
  `str` -> `"categorical"`

description : str
  Optional argument that specifies a short description of the parameter
  and should fit into half a line.
  Defaults to empty string.

info : str
  Optional argument that provides more information than the description.
  Will be broken into multiple lines if necessary.
  Defaults to empty string.
\end{verbatim}

\begin{enumerate}

\setcounter{enumi}{2}
\tightlist
\item
  Create an instance of \texttt{Solver} using the following arguments:
\end{enumerate}

\begin{verbatim}
name : str
  Required argument that specifies the name of the solver.

params : list(Parameter)
  Optional argument that defines all the tuning parameters.
  Defaults to empty list.

paths : list(str)
  Optional argument that specifies all paths of the solver
  which are needed for calling it and will be added to the workspace.
  Defaults to empty list.

preparation : str
  Optional argument that is valid MATLAB code, which is needed
  to prepare all input options of the solver.
  Defaults to empty string.

call : str
  Optional argument that is valid MATLAB code, which calls the solver
  and processes the output as needed.
  Defaults to empty string.

tune_cases : list(list(Parameter))
  Optional argument that specifies more predefined parameter options
  that can be selected when calling the wrapper.
  Defaults to empty list.
\end{verbatim}

\begin{enumerate}

\setcounter{enumi}{3}
\tightlist
\item
  Call the desired wrapper or tuner writing method on the defined solver
  instance:
\end{enumerate}

\begin{verbatim}
Solver.write_wrapper(save_path)
  save_path : str
    Optional argument that specifies path to which the wrapper
    will be saved.
    Defaults to path that the script is executed from.

Solver.write_tuner(save_path)
  save_path : str
    Optional argument that specifies path to which the tuner
    will be saved.
    Defaults to path that the script is executed from.
\end{verbatim}

To make structural changes to all wrappers and tuners edit the
\texttt{WRAPPER\_TEMPLATE.txt} or \texttt{TUNER\_TEMPLATE.txt}. Make
sure to not alter the keywords, unless you intend to edit them in
\texttt{data.py} too. All solver scripts that require this change then
need to be run again.

See \texttt{example.py} for a showcase of an ideal structure for such a
script. The files \texttt{wSOLVER.m} and \texttt{tSOLVER.m} were
generated with this script.

Author: Maximilian Stollmayer, maximilian.stollmayer@univie.ac.at

\end{document}
